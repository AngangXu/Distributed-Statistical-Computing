\section{实例分析:基于 MapReduce
的电影推荐系统}\label{ux5b9eux4f8bux5206ux6790ux57faux4e8e-mapreduce-ux7684ux7535ux5f71ux63a8ux8350ux7cfbux7edf}

\subsection{准备知识}\label{ux51c6ux5907ux77e5ux8bc6}

\begin{itemize}
\itemsep1pt\parskip0pt\parsep0pt
\item
  推荐系统
\item
  Python
\item
  Mapreduce
\end{itemize}

推荐系统在互联网领域有着广泛的运用,本案例实现了对大量电影评论数据的整合处理,并根据电影间的
相关程度,根据用户观影历史向用户推荐相关度较高的电影。

\subsection{数据准备}\label{ux6570ux636eux51c6ux5907}

本文所用数据来自
\href{http://grouplens.org/datasets/movielens/}{GroupLens 网站},包含
1000 名用 户对 1700 部电影的 100000 条 评论(1998
年),为方便处理首先对数据进行了简单预处理。10 万条数
据中,每条数据由用户编号、电影名称、评分组成。

\subsection{建立模型}\label{ux5efaux7acbux6a21ux578b}

针对数据特点,电 影的推荐系统设计如下:

\begin{itemize}
\item
  对每一对电影 A 和 B,对所有评论过这两部电影的用户及评分数据进行聚集;
\item
  使用评分数据计算电影 A 与 B 的相关系数;
\item
  对所有看过某电影的用户,我们可以向其推荐与其相关系数最高的几部电影。
\end{itemize}

该推荐系统的简要 MapReduce 步骤如下。

\subsection{把评分数据按用户进行聚集}\label{ux628aux8bc4ux5206ux6570ux636eux6309ux7528ux6237ux8fdbux884cux805aux96c6}

原始数据(user\textbar{}movie\textbar{}rating)可视为 Key 为用户 ID,Value
为电影名和评分的 Key-Value 对,因此 Map 阶段不需要过多操作,可以直接使用
cat 函数作为 Mapper, 将 Key-Value 对传递给 Reducer。而 Reducer
则根据用户 ID,将同一用户评分过的所
有电影名和评分进行汇集。最后将同一用户的所有评
分数据储存在同一行,输出数据 共 1000
行。此处通过分行将不同用户的评分区分开来,因此用户 ID 可以在后续分析
中舍弃。

Mapper: cat 函数

Reducer:

\begin{lstlisting}
#! /usr/bin/python3

import sys
from collections import defaultdict

#reduce process
dict=defaultdict(list)
while True:
    try:
        line=sys.stdin.readline()
        if not line:
            break
        user_id, item_id, rating = line.split('|')
        dict[user_id].append((item_id,float(rating)))
    except:
        continue
sys.stdin.close()

with open('ratings1.csv','w') as f:
     for k in dict.keys():
         for j in dict[k]:


f.write(str(j[0])+'*'+str(j[1])+'|')
f.write('\n')

 #standard output to HDFS
 #use different separator for line split in next stage
for k in dict.keys():
    for j in dict[k]:
        print(str(j[0])+'*'+str(j[1])+'|',end="")
    print('\n')
\end{lstlisting}

其中reduce 代码中,使用了 defaultdict 类型对 Key-Value
型数据进行储存。相对于普通的字典型数 据,defaultdict
可以自动设定默认值,可以方便地储存本文的数据类 型。在读入数据时,使用了
try
函数防止错误格式的数据输入,当某一行数据格式有误时,自动从下一行继续读入。为了方便第二阶段
MapReduce 对数据进行分隔,此处 reduce
的输出使用了*和\textbar{}作为分隔符。

模拟 Hadoop 模式调试:

\begin{lstlisting}
cat ratings.csv | sort -t $'|' -k 1,1 | ./reduce1.pyHadoop
\end{lstlisting}

集群运行:

\begin{lstlisting}
hadoop jar /home/dmc/hadoop/share/hadoop/tools/lib/hadoop-streaming-2.6.0.jar \
   -D stream.map.output.field.separator=\| \
   -file /home/ruc15/chen_sicong/MR/reduce1.py \
   -input chen_sicong/ratings.csv \
   -output chen_sicong/output \
   -mapper "/bin/cat" \
   -reducer "reduce1.py"
\end{lstlisting}

Hadoop 集群模式下运行时,需要事先把数据文件上传至 HDFS,由于 mapper 的 输
出不是默认的逗号分隔,需要使用-D 指令进行设置。使用-file 指令将 Reducer
代码 分发到各个 节点,以解决``找不到执行程序''的错误。

\subsection{计算每两部电影之间的皮尔逊相关系数:}\label{ux8ba1ux7b97ux6bcfux4e24ux90e8ux7535ux5f71ux4e4bux95f4ux7684ux76aeux5c14ux900aux76f8ux5173ux7cfbux6570}

第二阶段的输入数据为第一阶段的输出数据,每一行是同一用户所有评论过的电影名和评分。在
Map 阶
段,首先将同一用户对各个电影的评分拆分开来,取其中所有的两两组合进行输出。这样,Mapper
输出的 数据为每个用户同时看过的电影两两组合及其评分。即 Key:(电影 1,电影
2),Value:(电影 1 评分,电 影 2 评分)。

Reduce 阶段根据 Key:(电影 1,电影 2)进行汇总,得到所有同时看过电影 1
和电影 2 的用户的评分。 然后,根据这些评分计算电影 1 和电影 2
的皮尔逊相关系数

Reducer
的输出为每两部电影之间的皮尔逊相关系数及评分人数,评分人数可以视为相关系数的可信度,
当评分人数过少时,相关系数未必可信。最后,根据相关系数可以对用户进行简单的电影推荐:当一个用
户看过某电影时,可以向其推荐与其相关程度较高的若干其他电影。

\begin{lstlisting}
#!/usr/bin/python3

from itertools import combinations
import sys

while True:

    try:
        line=sys.stdin.readline()9. if not line:
            break
        line=line.strip()
        values=line.split('|')
        #combinations(values,2) get all the combinations of 2 films

   for item1, item2 in combinations(values,2):
       #check if the items are empty

       if len(item1)>1 and len(item2)>1:
           item1,item2=item1.split('*'),item2.split('*')
           print((item1[0],item2[0]),end='')
           print('|',end='')
           print(item1[1]+','+item2[1])

           #output of map2: (film1,film2)|(score of film1,score of film2)

except:
    continue
\end{lstlisting}

Hadoop 集群运行:

\begin{lstlisting}
hadoop jar /home/dmc/hadoop/share/hadoop/tools/lib/hadoop-streaming-2.6.0.jar \
-D stream.map.output.field.separator=\| \
-file /home/ruc15/chen_sicong/MR/map2.py \
-file /home/ruc15/chen_sicong/MR/reduce2.py \
-input chen_sicong/ratings1.csv \
-output chen_sicong/output \
-mapper "map2.py" \
-reducer "reduce2.py"
\end{lstlisting}

根据两次 mapreduce 计算,我们得出了任意两部电影之间的相关系数。这样,当某
一用户看过电影 A
时,我们可以对其推荐相关程度最大的几部电影。一个简单的电影
推荐系统得以实现。在使用时应该将
相关系数与评价人数结合起来进行考虑,如果评
价人数过少,即使相关系数较高,结果也未必可信。本案
例实现的基于内容过滤的推荐系统简单快速,但不能为用户发现新的感兴趣的商品,只能发现和用户已有
兴趣相似的商品。在后续的改进中,可以建立更准确有效的推荐系统。

(感谢中国人民大学陈思聪提供素材和案例。)
