\section{实例分析:分层随机抽样}\label{ux5b9eux4f8bux5206ux6790ux5206ux5c42ux968fux673aux62bdux6837}

\subsection{准备知识}\label{ux51c6ux5907ux77e5ux8bc6}

\begin{itemize}
\itemsep1pt\parskip0pt\parsep0pt
\item
  分层抽样基本知识
\item
  R
\item
  Hadoop Streaming
\end{itemize}

\subsection{数据生成}\label{ux6570ux636eux751fux6210}

本次案例准备用Hadoop程序,实现对随机生成的总体数据进行分层随机抽样,然后计算总体均值的估计
值。总体数据由\lstinline|map|函数生成,没有保存到txt文档。

\begin{lstlisting}
	library(sampling)
	a <- runif(1000, 1, 10)
	b <- rnorm(2000, 2.5, 5)
	c <- rexp(1500, 0.5)
	t <- c(a,b,c)
\end{lstlisting}

其中a层为第一层,是一个包含1000个个体的服从(1,10)之间均匀分布的子总体;b层为第二层,是
一个均值为2.5,标准差为5,包含2000个个体的服从正态分布的子总体;c层为第三层,是一个包含
1500个个体的、均值为2的服从指数分布的子总体。

样本量为450,在第一层中抽取100个样本,在第二层中抽取200个样本,在第三层中抽取150个样本。由
于无法在Mapper和Reducer中分开计算标准差,无法使用抽样中变异系数对估计效果进行评价,只在
Mapper中计算总体均值真值和简单随机抽样得到均值的估计值,然后与分层随机抽样得到的估计值做直
观上比较。

\subsection{Mapper和Reducer设计}\label{mapperux548creducerux8bbeux8ba1}

\begin{itemize}
\item
  Mapper的任务是在每层中用简单随机的方式抽取需要的样本量,然后计算样本的均值,最后将均值传
  到Reducer中。本部分完整代码见\lstinline!01-stratified-sampling/xymap.R!文件。
\item
  Reducer的任务是用得到的均值加权平均得到总体均值的估计。本部分完整代码见
  \lstinline!01-stratified-sampling/xyred.R!文件。
\end{itemize}

\subsection{结果}\label{ux7ed3ux679c}

运行程序后得到分层随机抽样对总体均值的估计值为3.063439。

在Mapper中添加输出真值和简单随机抽样得到的估计值的命令,得到总体均值真值为3.106485,简单随
机抽样估计值为3.008685。

直观上,分层随机抽样的估计精度高于了简单随机抽样,但是只是进行了一次抽样,也没有比较不同样
本量,结果比较随机,需要进行较大的改进。
