\section{实例分析:2015年上映电影的聚类分析}\label{ux5b9eux4f8bux5206ux67902015ux5e74ux4e0aux6620ux7535ux5f71ux7684ux805aux7c7bux5206ux6790}

\subsection{准备知识}\label{ux51c6ux5907ux77e5ux8bc6}

\begin{itemize}
\itemsep1pt\parskip0pt\parsep0pt
\item
  K均值聚类算法
\item
  R 和 Python
\item
  Hadoop Streaming
\end{itemize}

\subsection{数据准备}\label{ux6570ux636eux51c6ux5907}

本次Hadoop实例是利用K均值聚类算法对豆瓣网上2015年上映的电影进行一次聚类分析,主要利用豆瓣
网上各个电影的评分和各个电影的评价人数这两个标准,对2015年所有上映的电影进行分类。数据的爬
取、清理使用Python语言,而K均值聚类模型建立的Map和Reduce,本实例同时采用了R和Python两种语言
进行编写。

\begin{itemize}
\item
  爬取数据:通过firefox网站访问豆瓣网,找到关于2015年中国内地票房年度总排行,点击后利用开
  发者中的查看器,查看网页脚本,找到需要提取信息所对应的脚本内容。启动新的爬虫项目,编写好
  items和spiders,爬取相应信息。并将爬取下来的信息另存到新的文件。
\item
  将爬取的信息利用Python语言进行清洗,去除掉评论人数过少导致评分缺失的数据,对数字以外的字
  符进行删除,并将新数据转化为数值型。
\item
  分别利用R和Python语言实现对2015年内地进上映电影进行K均值聚类,将实现过程分别写入Mapper和
  Reducer。
\end{itemize}

\subsection{建立K均值聚类模型}\label{ux5efaux7acbkux5747ux503cux805aux7c7bux6a21ux578b}

初始的数据集较为杂乱,我们需要重新对数据进行清洗和整合,按行提取数据后首先删除数据中的除数
字外的信息。并将数据格式转化为数值型。

\begin{lstlisting}
#! /usr/bin/python3
# -*- coding: utf-8 -*-

import numpy as np
from numpy import *
import pandas as pd
from pandas import *
import sys
from itertools import islice


df = open('kmeans.csv','r',encoding = 'utf-8')
lines = df.readlines()

rate = []
number = []

for line in lines:
part=line.split(',')
c1 =part[0]
rate.append(c1)
c2=part[1][1:-5]
number.append(c2)

rate = rate[1:]
number = number[1:]

for j in range(0,len(rate)):
rate[j] = float(rate[j])

for a in range(0,len(number)):
number[a] = int(number[a])

data = [rate] +[number]
data = transpose(np.array(data))
data = DataFrame(data, columns = ['rate', 'number'])

data.to_csv('clean.txt',index=False)
\end{lstlisting}

Hadoop的MapReduce技术,Mapper主要用于实现读取数据完输入数据并完成数据分割后开始运行。每个
分割后的切片数据都会以键值对数据进行输出。

\begin{lstlisting}
#! /usr/bin/env Rscript

input <- file("stdin","r")

A<-c()
B<-c()

while(length(currentLine <- readLines(input,n=1,warn=FALSE))>0)
{
    fields <- unlist(strsplit(currentLine,","))
    a=as.double(fields[1])
    b=as.double(fields[2])
    A=c(A,a)
    B=c(B,b)
}
mydata <- data.frame(A,B)

scale01=function(x){
    ncol=dim(x)[2]
    nrow=dim(x)[1]
    new=matrix(0,nrow,ncol)
    for(i in 1:ncol)
    {max=max(x[,i])
        min=min(x[,i])
        for(j in 1:nrow)
        {new[j,i]=(x[j,i]-min)/(max-min)}
    }
    new
}

datanorm=scale01(mydata)
print(datanorm,stdout())
close(input)
\end{lstlisting}

而Reducer在接收Map阶段的输出数据,该数据以键为基础进行分组。Reducer用于对数据进行聚集,接
受Map的数据后产生键值,以作为每个分组的数值。从K均值聚类的流程来看,我们需要对数据进行标准
化处理,将处理后的数据变为0-1之间分布的数值。然后再建立模型。

在R版本的Reducer中引入R包e1071,利用cmeans函数分析数据,将其进行K均值聚类分析。将聚类数目
设置为4,迭代数目设为20次。

\begin{lstlisting}
#! /usr/bin/env Rscript

input <- file("stdin","r")
A<-c()
B<-c()
while(length(currentLine<-readLines(input,n=1,warn=FALSE))>0)
{
    fields=unlist(strsplit(currentLine," "))
    a=as.double(fields[2])
    b=as.double(fields[3])
    A=c(A,a)
    B=c(B,b)
}

mydata=data.frame(A,B)
mydata=na.omit(mydata)
data=as.matrix(mydata)

library(e1071)
results=cmeans(data,centers=4,iter.max=20,verbose=TRUE,method="cmeans",m=2)
print(results)

close(input)
\end{lstlisting}

在Python版本Reducer的该函数为给定数据集构建一个包含K个随机质心的集合,随机质心必须要在整个
数据集的边界内,这可以通过找到数据集每一维的最小和最大值来完成。然后随机生成0到1之间的随机
数,并通过取值范围和最小值,以便确保随机点在数据的边界之内:

\begin{lstlisting}
def randCent(dataSet, k):
      n = shape(dataSet)[1]
      centroids = mat(np.zeros((k,n)))
      for j  in range(n):
           minJ = min(dataSet[:, j])
           rangeJ = float(max(dataSet[:,j]) - minJ)
           centroids[:,j] = minJ + rangeJ * random.rand(k, 1)
      return centroids
\end{lstlisting}

最后定义K均值算法的函数。一开始确定数据集中数据点的总数,然后创建一个矩阵来储存每个点的簇分配结果。簇分类结果有两列。一列记录簇的索引值,第二列储存误差。这里的误差指的是当前点到簇质心的距离。按照此种方式,反复迭代,知道所有数据点的簇分配结果不会改变为止。

\begin{lstlisting}
def kMeans(dataSet, k, distMeas=distEclud,createCent=randCent):
    m = shape(dataSet)[0]
    clusterAssment = mat(zeros((m,2)))
    centroids = createCent(dataSet,k)
    clusterChanged = True
    while clusterChanged:
      clusterChanged = False
          for i in range(m):
              minDist = inf; minIndex = -1
              for j in range(k):
                  distJI = distMeas(centroids[j,:],dataSet[i,:])
          if distJI < minDist:
             minDist = distJI; minIndex = j
          if clusterAssment[i,0] != minIndex: clusterChanged = True
          clusterAssment[i,:] = minIndex,minDist**2
          print centroids
          for cent in range(k):
          ptsInClust = dataSet[nonzero(clusterAssment[:,0].A==cent)[0]]
              centroids[cent,:] = mean(ptsInClust,axis=0)
      return centroids,clusterAssment
\end{lstlisting}

(感谢中国人民大学陈晞提供素材和案例。)
