\section{判别分析:气象因素对雾霾影响分析}\label{ux5b9eux4f8bux5206ux6790ux57faux4e8eux5224ux522bux5206ux6790ux7684ux6c14ux8c61ux56e0ux7d20ux5bf9ux96feux973eux7684ux5f71ux54cd}

\subsection{准备知识}\label{ux51c6ux5907ux77e5ux8bc6}

\begin{itemize}
\itemsep1pt\parskip0pt\parsep0pt
\item
  判别分析
\item
  R
\item
  Hadoop Streaming
\end{itemize}

\subsection{研究背景}\label{ux7814ux7a76ux80ccux666f}

众所周知,雾霾指数(AQI)由二氧化硫、二氧化氮、PM10、PM2.5、一氧化碳和臭氧这六项污染物计算
所得。然而我们生活中经常熟知的气象因素,比如平均气温、平均水汽压、平均相对湿度、平均风速和
日照时数,是否会对雾霾有影响呢?那么简单的基于这些因素通过距离判别法对是否雾霾进行分析预测
是否可行呢?本文会通过Hadoop对此进行简单探究。

\subsection{研究思路}\label{ux7814ux7a76ux601dux8def}

本案例基于北京市2013年10月到2015年6月的日气象数据对是否雾霾进行判别分析。数据的自变量来自
于中国气象数据网的中国地面国际交换站气候资料日值数据集(站点为北京站),主要有平均气温、平
均水汽压、平均相对湿度、平均风速和日照时数。数据的因变量是根据AQI指数得出的是否雾霾指标
\lstinline|rank|(0、1变量)。研究思路是,将所有的样本分成训练样本集(所有样本去掉最后三十天样本)和测试
样本集(最后三十天样本),对训练样本集进行Hadoop下的判别分析建模,最后用测试样本集对模型进
行测试,计算错判率,分析结果。

\subsection{建立判别分析模型}\label{ux5efaux7acbux5224ux522bux5206ux6790ux6a21ux578b}

距离判别的重点是求得每个类别中各个变量的均值作为类中心,然后基于每个类别中各个变量的均值和
方差求得观测数据点到各类别中心的马氏距离,根据距离的远近进行判别。对于训练样本集中每类的各
个变量的均值和方差的计算我们用\lstinline!mapper!和\lstinline!reducer!进行分布式计算。下面我们以0类的第一个变量
trainx10为例进行展示。

首先用R将trainx10数据提出保存为trainx10.txt。然后我们用\lstinline!mapper!和\lstinline!reducer!计算均值。

\begin{lstlisting}
	R> pdata <- read.csv('climate.csv', header=T)
	R> trainx10 <- matrix((subset(pdata[,1],pdata[,6]==0)))
	R> write.table(trainx10,file="d:/trainx10.txt",sep=",",row.names=F,col.names=F)
\end{lstlisting}

\lstinline|Mapper|函数计算每一块数据的均值,\lstinline!reducer!函数计算由\lstinline!mapper!传递来的所有数据的均值。经检验与数
据真实的均值一致。

最后根据第一组\lstinline|mapper|和\lstinline|reducer|函数中得到的均值,再写一组\lstinline!mapper!和\lstinline!reducer!,这次\lstinline!mapper!函数需要计算每一块中每个数据与均值
的差的平方和,\lstinline!reducer!为由\lstinline!mapper!传递过来的所有数据的和除以n-1。最后所得值就是方差,
经验证与数据真实的方差一致。可以编写多组\lstinline|mapper|和\lstinline|reducer|函数来实现并行。

0类的其他变量的均值方差的计算与x10类似,1类的所有变量的均值方差与0类的计算类似。最后分别计算测试
样本数据与0类和1类的马氏距离,离哪类比较近就判为哪一类,可以得到测试样本的预测判别变量
prerank。

\subsubsection{Mapper函数}\label{mapperux51fdux6570}

计算0类中的x1(airtemperature)变量的均值的\lstinline!mapper!函数如下所示,计算每一块数据的均值。计算0类中的x1(airtemperature)变量方差的\lstinline!mapper!函数与之类似。

\begin{lstlisting}
	#! /usr/bin/env Rscript
	options(warn=-1)
	sink("/dev/null")
	input<-file("stdin","r")
	while(length(currentLine<-readLines(input, n=1, warn=FALSE)) > 0)
	{
	    fields<-unlist(strsplit(currentLine, ","))
	    data<-sum(as.numeric(fields))
	    sink()
	    cat(data,"\n", sep="\t")
	    sink("/dev/null")
	}
	close(input)
\end{lstlisting}

\subsubsection{Reducer函数}\label{reducerux51fdux6570}

计算0类中的x1(airtemperature)变量的均值的\lstinline!reducer!函数如下所示,用来计算由\lstinline!mapper!传递到\lstinline!reducer!中所有数据的均值。计算0类中的x1(airtemperature)变量的方差的\lstinline|reducer|函数与之类似。

\begin{lstlisting}
	#! /usr/bin/env Rscript
	options(warn=-1)
	sink("/dev/null")
	input <- file("stdin","r")
	data <- 0
	while(length(currentLine<-readLines(input, n=1, warn=FALSE)) > 0)
	{
	    fields <- unlist(strsplit(currentLine, "\t"))
	    data <- data+as.numeric(fields)
	}
	x10bar <- data / 305
	sink()
	cat(x10bar, "\n", sep="\t")
	sink("/dev/null")
	close(input)
\end{lstlisting}

\subsection{结论}\label{ux7ed3ux8bba}

由本文的研究可知平均气温、平均水汽压、平均相对湿度、平均风速和日照时数对雾霾有影响。简单的
基于这些因素通过距离判别法对是否雾霾进行分析预测的错判率为0.3,可以用此方法进行预测。
