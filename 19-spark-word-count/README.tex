\section{Spark实例:通过Word Count了解Spark工作流
  程
}\label{ux5b9eux4f8bux5206ux6790ux901aux8fc7word-countux4e86ux89e3sparkux5de5ux4f5cux6d41ux7a0b}

\subsection{准备知识}\label{ux51c6ux5907ux77e5ux8bc6}

\begin{itemize}
\itemsep1pt\parskip0pt\parsep0pt
\item
  Spark
\end{itemize}

\subsection{基本步骤}\label{ux57faux672cux6b65ux9aa4}

编写Spark应用与Hadoop类似。将需要的代码写入一个惰性求值的driver
program中,通过一个action, driver
program被分发到集群上,由各个RDD分区上的worker来执行。然后结果会被发送回driver
program进行聚合或编译。本质上,驱动程序创建一个或多个RDD,调用操作来转换RDD,然后调用动作
处理被转换后的RDD。

这些步骤大体如下:

\begin{itemize}
\item
  定义一个或多个RDD,可以通过获取存储在磁盘上的数据(HDFS,HBase,Local
  Disk),并行化内存
  中的某些集合,转换(transform)一个已存在的RDD,或者,缓存或保存。
\item
  通过传递一个闭包(函数)给RDD上的每个元素来调用RDD上的操作。Spark提供了除了Map和Reduce的
  80多种高级操作。
\item
  使用结果RDD的动作(action)(如count、collect、save等)。动作将会启动集群上的计算。
\item
  当Spark在一个worker上运行闭包时,闭包中用到的所有变量都会被拷贝到节点上,但是由闭包的局
  部作用域来维护。Spark提供了两种类型的共享变量,这些变量可以按照限定的方式被所有worker访
  问。广播变量会被分发给所有worker,但是是只读的。累加器这种变量,worker可以使用关联操作来
  ``加'',通常用作计数器。
\end{itemize}

下面我们来简略描述下Spark的执行。

本质上,Spark应用作为独立的进程运行,由驱动程序中的SparkContext协调。这个context将会连接到
一些集群管理者(如YARN),这些管理者分配系统资源。集群上的每个worker由执行者(executor)管
理,执行者反过来由SparkContext管理。执行者管理计算、存储,还有每台机器上的缓存。

重点要记住的是应用代码由驱动程序发送给执行者,执行者指定context和要运行的任务。执行者与驱
动程序通信进行数据分享或者交互。驱动程序是Spark作业的主要参与者,因此需要与集群处于相同的
网络。这与Hadoop代码不同,Hadoop中你可以在任意位置提交作业给JobTracker,JobTracker处理集群
上的执行。

\subsection{Word Count 演示}\label{word-count-ux6f14ux793a}

为了演示“word
count”的功能,我们在计算机路径下存储了一部小说《福尔摩斯全集》。

首先,在命令行中输入以下命令:

\begin{lstlisting}
	$ pyspark
\end{lstlisting}

PySpark将会自动使用本地Spark配置创建一个SparkContext。你可以通过sc变量来访问它。

下面我们来创建第一个RDD

\begin{lstlisting}
	Pyspark> text = sc.textFile("SherlockHolmes.txt")
	Pyspark> print text
\end{lstlisting}

由于我们访问的是本地的数据文件,所以没有必要使用addFile函数。textFile方法将数据文件加载到
一个RDD命名文本。使用print命令查看RDD,可以发现这是一个MapPartitionsRDD。注意,文件路径是
相对于当前工作目录的一个相对路径,在上一段命令中我们没有输入文件的路径是因为我们是在数据文
件所在的目录下打开Pyspark的。我们转换下这个RDD,来进行分布式计算的“hello
world”:

\begin{lstlisting}
	Pyspark> from operator import add
	Pyspark> def acf(text):
		       return text.split()
	Pyspark> words = text.flatMap(acf)
	Pyspark> print words
\end{lstlisting}

我们首先引入了add这个函数,作为加法的闭包来使用。首先我们要做的是把文本拆分为单词。我们创
建了一个acf函数,参数是文本片段,返回根据空格拆分的单词列表。然后我们通过给flatMap操
作符传递acf闭包对textRDD进行变换创建了一个wordsRDD。words是个PythonRDD,但是执行本应
该立即进行。显然,我们还没有把整个数据集拆分为单词列表。

如果你曾使用MapReduce做过Hadoop版的“Word Count”,你应该知道下一步是将每个单词映射到一个键值
对,其中键是单词,值是1,然后使用reducer计算每个键的1总数。

首先,我们定义一个mapper

\begin{lstlisting}
	Pyspark> def chag(x):
			return (x,1)
	Pyspark> wc = words.map(chag)
	Pyspark> print wc.toDebugString()
\end{lstlisting}

这行代码将会把chag映射到每个单词。因此,每个x都是一个单词,每个单词都会被匿名闭包转换为
元组(word,
1)。为了查看转换关系,我们使用toDebugString方法来查看RDD是怎么被转换的。

接下来可以使用reduceByKey动作进行字数统计,然后把统计结果写到磁盘。

\begin{lstlisting}
	Pyspark> counts = wc.reduceByKey(add)
	Pyspark> counts.saveAsTextFile("wc_result")
\end{lstlisting}

一旦我们最终调用了\texttt{saveAsTextFile}动作,这个分布式应用就开始执行了,在应用“跨集群”(或者你
本机的很多进程)运行时,你应该可以看到很多INFO语句。如果退出解释器,你可以看到当前工作目录
下有个“wc\_result”目录。

每个part文件都代表你本机上的进程计算得到的被保持到磁盘上的最终RDD。如果对一个part文件运行
head命令,可以看到字数统计元组。注意这些键没有像Hadoop一样被排序(因为Hadoop中Map和
Reduce任务中有个必要的打乱和排序阶段),但是,能保证每个单词在所有文件中只出现一次,因为我
们使用了reduceByKey操作符。
